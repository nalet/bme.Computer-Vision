% --------------------------------------------------------------
% This is all preamble stuff that you don't have to worry about.
% Head down to where it says "Start here"
% --------------------------------------------------------------

\documentclass[12pt]{article}

\usepackage[margin=1in]{geometry}
\usepackage{amsmath,amsthm,amssymb}
\usepackage{graphicx} %This allows to include eps figures
\usepackage{subcaption}
\usepackage[section]{placeins}
\usepackage{layout}
\usepackage{etoolbox}
\usepackage{mathabx}
% This is to include code
\usepackage{listings}
\usepackage{xcolor}
\definecolor{dkgreen}{rgb}{0,0.6,0}
\definecolor{gray}{rgb}{0.5,0.5,0.5}
\definecolor{mauve}{rgb}{0.58,0,0.82}
\lstdefinestyle{Python}{
    language        = Python,
    basicstyle      = \ttfamily,
    keywordstyle    = \color{blue},
    keywordstyle    = [2] \color{teal}, % just to check that it works
    stringstyle     = \color{green},
    commentstyle    = \color{red}\ttfamily
}

\newcommand{\N}{\mathbb{N}}
\newcommand{\R}{\mathbb{R}}
\newcommand{\X}{\mathbb{X}}
\newcommand{\Z}{\mathbb{Z}}
\def\doubleunderline#1{\underline{\underline{#1}}}

\newenvironment{theorem}[2][Theorem]{\begin{trivlist}
\item[\hskip \labelsep {\bfseries #1}\hskip \labelsep {\bfseries #2.}]}{\end{trivlist}}
\newenvironment{lemma}[2][Lemma]{\begin{trivlist}
\item[\hskip \labelsep {\bfseries #1}\hskip \labelsep {\bfseries #2.}]}{\end{trivlist}}
\newenvironment{exercise}[2][Exercise]{\begin{trivlist}
\item[\hskip \labelsep {\bfseries #1}\hskip \labelsep {\bfseries #2.}]}{\end{trivlist}}
\newenvironment{reflection}[2][Reflection]{\begin{trivlist}
\item[\hskip \labelsep {\bfseries #1}\hskip \labelsep {\bfseries #2.}]}{\end{trivlist}}
\newenvironment{proposition}[2][Proposition]{\begin{trivlist}
\item[\hskip \labelsep {\bfseries #1}\hskip \labelsep {\bfseries #2.}]}{\end{trivlist}}
\newenvironment{corollary}[2][Corollary]{\begin{trivlist}
\item[\hskip \labelsep {\bfseries #1}\hskip \labelsep {\bfseries #2.}]}{\end{trivlist}}

\begin{document}

% --------------------------------------------------------------
%                         Start here
% --------------------------------------------------------------

%\renewcommand{\qedsymbol}{\filledbox}

\title{Assignment 1 - Image Deblurring}%replace X with the appropriate number
\author{Nalet Meinen \\ %replace with your name
Computer Vision
}

\maketitle

\section{Finite difference approximation of the objective \\
         function $E$}

The deblurring problem can be written as the following optimization
\begin{align}
    \hat{u} &= \mathrm{arg} \underbrace{min}_{u} E[u] \\
    E[u]    &= |g - u * k|^2 + \lambda R[u]
\end{align}

\noindent where the term $ |g - u * k|^2 $ equals to:

\begin{align}
    |g - u * k|^2 = \sum_{i=0}^{m-1} \sum_{j=0}^{n-1} \Big| g[i,j] - \sum_{p=0}^{1} \sum_{q=0}^{1} k[p,q] u[i-p + 1,j - q + 1] \Big|^2_2
\end{align}

\noindent where $g$ is the blurred image, $k$ is the kernel and $u$ should be our result, the deblurred image.

\noindent The kernel used for this assignment is:

\begin{align}
    k_1 =   \begin{bmatrix}
                \frac{1}{2} & 0 \\
                \frac{1}{2} & 0 \\
            \end{bmatrix}
\end{align}

\subsection{Derive the corresponding discretization for the anisotropic prior}

We want to discretize the anisotropic prior:

\begin{align}
    R[u] = | \nabla u |_1 = \sum_{i=0}^{m} \sum_{j=0}^{n} | \nabla u[i,j] |_1
\end{align}

\noindent the discretized version is similar but takes the absolute value.

\begin{align}
    \resizebox{1.0\hsize}{!}{$
        R[u] = \sum_{i=0}^{m-1} \sum_{j=0}^{n-1} |u[i + 1,j] - u[i,j]| + |u[i, j + 1] - u[i,j]| + \sum_{j=0}^{n-1} |u[m, j + 1] - u[m,j]| + \sum_{i=0}^{m-1} |u[i + 1,n] - u[i,n]|
    $}
\end{align}


\section{Calculation of the exact gradient of the \\
         discretized $E$}

\subsection{$\lambda = 0$}

With $\lambda = 0$ the problem reduces to:
\begin{align*}
    E[u] &= |g - u * k|^2 \\
    E[u] &= \sum_{i=0}^{m-1} \sum_{j=0}^{n-1} \Big| g[i,j] - \sum_{p=0}^{1} \sum_{q=0}^{1} k[p,q] u[i-p + 1,j - q + 1] \Big|^2_2
\end{align*}

As the kernel has only four entries, we get rid of the summation:
{\tiny  
	\setlength{\abovedisplayskip}{6pt}
	\setlength{\belowdisplayskip}{\abovedisplayskip}
	\setlength{\abovedisplayshortskip}{0pt}
	\setlength{\belowdisplayshortskip}{3pt}
    \begin{align*}
                                                                k   &= \begin{bmatrix}
                                                                            \frac{1}{2} & 0 \\
                                                                            \frac{1}{2} & 0 \\
                                                                        \end{bmatrix} \\
        \sum_{p=0}^{1} \sum_{q=0}^{1} k[p,q] u[i-p + 1,j - q + 1]   &= \frac{1}{2} u[i-0 + 1,j - 0 + 1] + 0 u[i-0 + 1,j - 1 + 1] + \frac{1}{2} u[i-1 + 1,j - 0 + 1] + 0 u[i-1 + 1,j - 1 + 1] \\
                                                                    &= \frac{1}{2} u[i-0 + 1,j - 0 + 1] + \frac{1}{2} u[i-1 + 1,j - 0 + 1] \\
                                                                    &= \underline{ \frac{1}{2} u[i + 1,j + 1] + \frac{1}{2} u[i, j + 1] } \\
    \end{align*}
}%


\begin{align*}
    \nabla E[u] &= \nabla \Big ( \sum_{i=0}^{m-1} \sum_{j=0}^{n-1} \Big| g[i,j] - \frac{1}{2} u[i + 1,j + 1] - \frac{1}{2} u[i, j + 1]  \Big|^2_2 \Big ) \\
    \nabla E[u] &= \nabla \Big ( \sum_{i=0}^{m-1} \sum_{j=0}^{n-1} \Big| g[i,j] - \frac{1}{2} u[i + 1,j + 1] - \frac{1}{2} u[i, j + 1]  \Big|^2_2 \Big ) \\
    \nabla E[u] &= \nabla \Big ( \sum_{i=0}^{m-1} \sum_{j=0}^{n-1} ( g[i,j] - \frac{1}{2} u[i + 1,j + 1] - \frac{1}{2} u[i, j + 1]  )^2 \Big ) \\
    \nabla E[u] &= \underline{ \sum_{i=0}^{m-1} \sum_{j=0}^{n-1} 2g[i,j] - u[i + 1,j + 1] - u[i, j + 1] } \\
\end{align*}

\noindent we finish here with the basis form where then we go further. 

Pixels influence each other, we have to look from all perspectives for $\nabla E[i,j]$: center pixel (i, j), northern pixel (i, j - 1), southern pixel (i, j + 1), western pixel  (i - 1, j), eastern pixel  (i + 1, j), north-western pixel (i - 1, j - 1), north-eastern pixel (i + 1, j - 1), south-western pixel (i - 1, j + 1), south-eastern pixel  (i + 1, j + 1).
{\tiny  
	\setlength{\abovedisplayskip}{6pt}
	\setlength{\belowdisplayskip}{\abovedisplayskip}
	\setlength{\abovedisplayshortskip}{0pt}
    \setlength{\belowdisplayshortskip}{3pt}
    \renewcommand{\arraystretch}{2.5}
    \begin{center}
        \begin{tabular}{ c c c c c c }
            $2g[i-1,j-1] - u[i,j] - u[i-1, j]$    & $+$ &     $2g[i,j-1] - u[i + 1,j] - u[i, j]$  & $+$ &     $2g[i+1,j-1] - u[i+2,j] - u[i+1, j]$  & $+$ \\ 
            $2g[i-1,j] - u[i,j + 1] - u[i-1, j + 1]$    & $+$ &     $2g[i,j] - u[i + 1,j + 1] - u[i, j + 1]$  & $+$ &     $2g[i+1,j] - u[i+2,j + 1] - u[i+1, j + 1]$  & $+$ \\  
            $2g[i-1,j+1] - u[i,j+2] - u[i-1, j+2]$    & $+$ &     $2g[i,j+1] - u[i + 1,j+2] - u[i, j+2]$  & $+$ &     $2g[i+1,j+1] - u[i+2,j+2] - u[i+1, j+2]$  &    
        \end{tabular}
    \end{center}
}%

As we only try to find the derivative for $\nabla E[i,j]$, most value indexes that don't have $i$ or $j$ will be zero. This leaves us with this term:
\begin{align*}
    \nabla E[i,j] =   & - u[i,j] + 2g[i,j-1] - u[i+1,j] - u[i,j] - u[i+2,j] \\
                    & - u[i+1,j] + 2g[i-1,j] - u[i,j+1] + 2g[i,j] - u[i,j+1] \\ 
                    & + 2g[i+1,j] - u[i,j+2] + 2g[i,j+1] - u[i,j+2] \\
    \nabla E[i,j] =   & \doubleunderline{- 2u[i,j] + 2g[i,j-1] - 2u[i+1,j] - u[i+2,j] } \\
                    & \doubleunderline{+ 2g[i-1,j] - 2u[i,j+1] + 2g[i,j] } \\ 
                    & \doubleunderline{+ 2g[i+1,j] - 2u[i,j+2] + 2g[i,j+1] }
\end{align*}

Lets define the behavior at the boundaries: We use the Dirichlet approach, where I set the function to a constant, in this case $0$.
We define our range as $\forall i \in [1,M-2], \forall j \in [1,N-2]$
\begin{center}
    \begin{tabular}{ c c c }
        $[i,0]$ & $\rightarrow$  & $0$ \\
        $[0,j]$ & $\rightarrow$  & $0$ \\
        $[i,N]$ & $\rightarrow$  & $0$ \\
        $[M,J]$ & $\rightarrow$  & $0$ \\
        $[i,N-1]$ & $\rightarrow$  & $0$ \\
        $[M-1,J]$ & $\rightarrow$  & $0$ \\
    \end{tabular}
\end{center}

Our loop will go from $1$ to $M-2$ or $N-2$.


\subsection{Gaussian prior}

The derivative of the data term stays the same, so we only have to look at the Gaussian prior which is:
{\scriptsize  
	\setlength{\abovedisplayskip}{6pt}
	\setlength{\belowdisplayskip}{\abovedisplayskip}
	\setlength{\abovedisplayshortskip}{0pt}
	\setlength{\belowdisplayshortskip}{3pt}
    \begin{align*}
        R[u] &= | \nabla u |^2_2 = \sum_{i=0}^{m} \sum_{j=0}^{n} | \nabla u[i,j] |^2_2 \\
        R[u] &= \sum_{i=0}^{m-1} \sum_{j=0}^{n-1} (u[i + 1,j] - u[i,j])^2 + (u[i, j + 1] - u[i,j])^2 + \sum_{j=0}^{n-1} (u[m, j + 1] - u[m,j])^2 + \sum_{i=0}^{m-1} (u[i + 1,n] - u[i,n])^2
    \end{align*}
}%

The derivate then would be:
{\tiny  
	\setlength{\abovedisplayskip}{6pt}
	\setlength{\belowdisplayskip}{\abovedisplayskip}
	\setlength{\abovedisplayshortskip}{0pt}
	\setlength{\belowdisplayshortskip}{3pt}
    \begin{align*}
        \nabla R[u] &= \sum_{i=0}^{m-1} \sum_{j=0}^{n-1} -2(u[i + 1,j] - u[i,j]) - 2(u[i, j + 1] - u[i,j]) + \sum_{j=0}^{n-1} - 2(u[m, j + 1] - u[m,j]) + \sum_{i=0}^{m-1} - 2(u[i + 1,n] - u[i,n]) \\
    \end{align*}
}%

For the iterative version this would result into this derivative:
{\scriptsize  
	\setlength{\abovedisplayskip}{6pt}
	\setlength{\belowdisplayskip}{\abovedisplayskip}
	\setlength{\abovedisplayshortskip}{0pt}
	\setlength{\belowdisplayshortskip}{3pt}
    \begin{align*}
        u_\nabla R[i,j] &= \nabla \Big( - 2(u[i + 1,j] - u[i,j]) - 2(u[i, j + 1] - u[i,j]) - 2(u[m, j + 1] - u[m,j]) - 2(u[i + 1,n] - u[i,n]) \Big) \\
        u_\nabla R[i,j] &= - 2(u[i + 1,j] - u[i,j]) - 2(u[i, j + 1] - u[i,j]) - 2(0 - u[m,j]) - 2(0 - u[i,n]) \\
        u_\nabla R[i,j] &= - 2(u[i + 1,j] - u[i,j]) - 2(u[i, j + 1] - u[i,j]) + 2u[m,j] + 2u[i,n] \\
    \end{align*}
}%

Complete Equation:
{\scriptsize  
	\setlength{\abovedisplayskip}{6pt}
	\setlength{\belowdisplayskip}{\abovedisplayskip}
	\setlength{\abovedisplayshortskip}{0pt}
	\setlength{\belowdisplayshortskip}{3pt}
    \begin{align*}
        \nabla E[i,j] =   & \doubleunderline{- 2u[i,j] + 2g[i,j-1] - 2u[i+1,j] - u[i+2,j] } \\
                          & \doubleunderline{+ 2g[i-1,j] - 2u[i,j+1] + 2g[i,j] } \\ 
                          & \doubleunderline{+ 2g[i+1,j] - 2u[i,j+2] + 2g[i,j+1] } \\
                          & \doubleunderline{+ \lambda (- 2(u[i + 1,j] - u[i,j]) - 2(u[i, j + 1] - u[i,j]) + 2u[m,j] + 2u[i,n]) }
    \end{align*}
}%
hint: $\forall i \in [1,M-2], \forall j \in [1,N-2]$ \\
The boundaries behave the same as before.

\subsection{Anisotropic Total Variation}
The derivative of the data term stays the same, so we only have to look at the anisotropic Gaussian prior which is:
{\scriptsize  
	\setlength{\abovedisplayskip}{6pt}
	\setlength{\belowdisplayskip}{\abovedisplayskip}
	\setlength{\abovedisplayshortskip}{0pt}
	\setlength{\belowdisplayshortskip}{3pt}
    \begin{align*}
        R[u] &= | \nabla u |_1 = \sum_{i=0}^{m} \sum_{j=0}^{n} | \nabla u[i,j] |_1 \\
        R[u] &= \sum_{i=0}^{m-1} \sum_{j=0}^{n-1} |u[i + 1,j] - u[i,j]| + |u[i, j + 1] - u[i,j]| + \sum_{j=0}^{n-1} |u[m, j + 1] - u[m,j]| + \sum_{i=0}^{m-1} |u[i + 1,n] - u[i,n]|
    \end{align*}
}%

{\scriptsize  
	\setlength{\abovedisplayskip}{6pt}
	\setlength{\belowdisplayskip}{\abovedisplayskip}
	\setlength{\abovedisplayshortskip}{0pt}
	\setlength{\belowdisplayshortskip}{3pt}
    \begin{align*}
        \nabla R[u] &= \sum_{i=0}^{m-1} \sum_{j=0}^{n-1} -\frac{u[i + 1,j] - u[i,j]}{|u[i + 1,j] - u[i,j]|} - \frac{u[i, j + 1] - u[i,j]}{|u[i, j + 1] - u[i,j]|} + \sum_{j=0}^{n-1} -\frac{u[m, j + 1] - u[m,j]}{|u[m, j + 1] - u[m,j]|} + \sum_{i=0}^{m-1} -\frac{u[i + 1,n] - u[i,n]}{|u[i + 1,n] - u[i,n]|} \\
    \end{align*}
}%

For the iterative version this would result into this derivative:
{\scriptsize  
	\setlength{\abovedisplayskip}{6pt}
	\setlength{\belowdisplayskip}{\abovedisplayskip}
	\setlength{\abovedisplayshortskip}{0pt}
	\setlength{\belowdisplayshortskip}{3pt}
    \begin{align*}
        \nabla R[i,j] &= \nabla \Bigg( -\frac{u[i + 1,j] - u[i,j]}{|u[i + 1,j] - u[i,j]|} - \frac{u[i, j + 1] - u[i,j]}{|u[i, j + 1] - u[i,j]|} -\frac{u[m, j + 1] - u[m,j]}{|u[m, j + 1] - u[m,j]|} -\frac{u[i + 1,n] - u[i,n]}{|u[i + 1,n] - u[i,n]|} \Bigg) \\
        \nabla R[i,j] &= -\frac{u[i + 1,j] - u[i,j]}{|u[i + 1,j] - u[i,j]|} - \frac{u[i, j + 1] - u[i,j]}{|u[i, j + 1] - u[i,j]|} -\frac{u[i + 1,n] - u[i,n]}{|u[i + 1,n] - u[i,n]|} \\
    \end{align*}
}%

Complete Equation:
{\scriptsize  
	\setlength{\abovedisplayskip}{6pt}
	\setlength{\belowdisplayskip}{\abovedisplayskip}
	\setlength{\abovedisplayshortskip}{0pt}
	\setlength{\belowdisplayshortskip}{3pt}
    \begin{align*}
        \nabla E[i,j] =   & \doubleunderline{- 2u[i,j] + 2g[i,j-1] - 2u[i+1,j] - u[i+2,j] } \\
        & \doubleunderline{+ 2g[i-1,j] - 2u[i,j+1] + 2g[i,j] } \\ 
        & \doubleunderline{+ 2g[i+1,j] - 2u[i,j+2] + 2g[i,j+1] } \\
        & \doubleunderline{+ \lambda \Bigg(-\frac{u[i + 1,j] - u[i,j]}{|u[i + 1,j] - u[i,j]|} - \frac{u[i, j + 1] - u[i,j]}{|u[i, j + 1] - u[i,j]|} -\frac{u[i + 1,n] - u[i,n]}{|u[i + 1,n] - u[i,n]|} \Bigg) }
    \end{align*}
}%
hint: $\forall i \in [1,M-2], \forall j \in [1,N-2]$ \\
The boundaries behave the same as before.
\end{document}