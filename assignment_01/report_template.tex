\documentclass{paper}

%\usepackage{times}
\usepackage{epsfig}
\usepackage{graphicx}
\usepackage{amsmath}
\usepackage{amssymb}
\usepackage{color}


% load package with ``framed'' and ``numbered'' option.
%\usepackage[framed,numbered,autolinebreaks,useliterate]{mcode}

% something NOT relevant to the usage of the package.
\setlength{\parindent}{0pt}
\setlength{\parskip}{18pt}






\usepackage[latin1]{inputenc} 
\usepackage[T1]{fontenc} 

\usepackage{listings} 

\newcommand{\equationinalign}[1]{%
  \multispan{2}%
  \hfill$\displaystyle{#1}$\hfill
  \ignorespaces
}

\lstset{% 
   language=Python, 
   basicstyle=\small\ttfamily, 
} 



\title{Assignment 1}



\author{Nalet Meinen\\13-463-955}
% //////////////////////////////////////////////////


\begin{document}



\maketitle


% Add figures:
%\begin{figure}[t]
%%\begin{center}
%\quad\quad   \includegraphics[width=1\linewidth]{ass2}
%%\end{center}
%
%\label{fig:performance}
%\end{figure}

\section*{Image blending}

\begin{enumerate}
\item \textbf{Exercise 1}

\begin{align*}
   \lVert \Delta u \rVert_2 \simeq \sum_{i,j} \sqrt{ u[i+1,j] - u[i,j]^2 + ( u[i,j + 1] - u[i,j] )^2 }
\end{align*}

\item \textbf{Exercise 2}

1)
\begin{align*}
   |u_C-g_C|^2_\Omega &= \sum_{i,j} \Omega[i,j] | u_C[i,j] - g_C[i,j] |^2_2 \\
   \frac{\partial}{\partial u[p,q]} |u_C-g_C|^2_\Omega &= \frac{\partial}{\partial u[p,q]} \sum_{i,j} \Omega[i,j] | u_C[i,j] - g_C[i,j] |^2_2
\end{align*}
\begin{align*}
   \equationinalign{\textrm{if} \; p = i \; \textrm{and} \; q = j \; \textrm{:}} \\
   \frac{\partial}{\partial u[p,q]} \sum_{i,j} \Omega[i,j] | u_C[i,j] - g_C[i,j] |^2_2 &= 2 \cdot \Omega[i,j] \cdot ( u_C[i,j] - g_C[i,j] ) \\
   \frac{\partial |u_C-g_C|^2_\Omega}{\partial u[i,j]} &= 2 \cdot \Omega[i,j] \cdot ( u_C[i,j] - g_C[i,j] )
\end{align*}

2)
\begin{align*}
   \frac{\partial |u_C-g_C|^2_\Omega}{\partial u[i,j]} &= \frac{\partial \tau[i,j]}{\partial u[i,j]} + \frac{\partial \tau[i - 1,j]}{\partial u[i,j]} + \frac{\partial \tau[i,j - 1]}{\partial u[i,j]} \\
   &= \frac{u_C[i,j] - u_C[i-1,j]}{\sqrt{(u_C[i,j] - u_C[i-1,j])^2 + (u_C[i-1,j+1] - u_C[i-1,j])^2}} + \\
   &\quad\; \frac{u_C[i,j] - u_C[i,j-1]}{\sqrt{(u_C[i+1,j-1] - u_C[i,j-1])^2 + (u_C[i,j] - u_C[i,j-1])^2}} + \\
   &\quad\; \frac{2 \cdot u_C[i,j] - u_C[i+1,j] - u_C[i,j+1]}{\sqrt{(u_C[i+1,j] - u_C[i,j])^2 + (u_C[i,j+1] - u_C[i,j])^2}} + \\
   & \textrm{where} \\
   \tau[i,j] &\doteq \sqrt{(u_C[i+1,j] - u_C[i,j])^2 + (u_C[i,j+1] - u_C[i-1,j])^2}
\end{align*}


\item \textbf{Implementation.} For each of the 3 solvers (gradient descent, Linearization+Gauss-Seidel, Linearization+SOR):

\begin{itemize}
\item Show images of the inputs
\item Show 5 images of the reconstruction as the method progresses iteration by iteration: The initial, the final image and 3 more images in between.
\item Show the energy against iteration time (we should see it decreasing over time).
\end{itemize}

\item \textbf{State which of the 3 solvers you choose. Show images obtained by very high, very low and manually-tuned (approximately optimal) $\lambda$.} In this section you should:

\begin{itemize}
\item Display 3 images with different $\lambda$: one with very low, one with very high and one with the manually-tuned (approximately optimal) $\lambda$.
\item Describe the effect of $\lambda$ on the solution.
\end{itemize}

\item \textbf{Image blending:} 
\begin{itemize}
\item Display your own image composition here along with the foreground, background and mask images.
\item Describe how you used or modified the code to create your image(s).
\end{itemize}


\end{enumerate}


 \end{document}
 
 